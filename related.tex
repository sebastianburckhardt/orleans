\section{Related Work}

 
Change propagation is semantically subtle. In sychronous models, time is explicit. Asynchronous models may make strong consistency guarantees based on some form of version tracking, or may have weaker consistency guarantees. For example, a so-called glitch means that an observer sees two observables A, B that have inconsistent state, meaning that the set of updates propagated to A is different from the set of updates propagated to B. Many reactive systems strive to eliminate glitches (e.g. using topological ordering of dependencies), but some embrace them.

Clearly, how to best balance consistency and performance is highly dependent on the architecture and workload. 
avoiding glitches in a distributed actor system like ours is likely to add significant latency overhead: updates are only partially ordered to begin with, and dependencies are detected dynamically. Also, for the applications we have in mind, it is typically preferrable to quickly display a glitchy result and then quickly correct it, than to generally wait longer. Consequently, we chose an algorithm that does not avoid glitches or causality violations categorically, but guarantees that they are ephemeral (\S\ref{sec:cp}). This tradeoff is quite similar to the variations of eventual consistency.
 

\subsection{Expressing Views.} Often, views can observe other views, creating a directed acyclic graph of dependencies. How to express such dependency graphs using recursive operators is a key question in the area of dataflow languages, functional reactive programming, and object-oriented frameworks such as Rx. In relational database systems, views are expressed by relational queries (which allows changes to be propagated incrementally). in Facebook's react.js, the application state is observed by a tree-structured virtual DOM, which is in turn observed by the browser's DOM. And in relational database systems, views are expressed by relational queries (which allows changes to be propagated incrementally)

 
 \subsection{Dependency Detection}. 

 

\cite{burckhardt-leijen-yi-sadowski-ball-OOPSLA11}
\cite{camil}


\cite{alive}
\cite{react}

\cite{elm} language for gui construction

\cite{statelines} inverse problem


\cite{orleans}
