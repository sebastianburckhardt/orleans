\section{Introduction}

emergence of client + cloud


increasingly interactive: browsers do not just fetch and display data, but run single-page application that communicate via websockets where server can push changes to clients efficiently. Games, collaborative editing, social features, online chats.

However, quite challenging to implement reactive services: elastic scaling, failure handling. Question: what programming models can help to write scalable reactive applications. 


Reactive programming is a term that can be meaningfully applied to many systems. At its core, reactivity is about expressing dependencies of an observer or view on some observed entity, and about a mechanism for propagating changes of the latter to the former. Thus, we consider a \emph{reactive programming model} to be a combination of (1) language support for expressing observers, or \emph{views}, and (2) a runtime implementation that propagates changes to views. The challenge is to strike the right balance between ease of use and performance, for the intended workload and architecture.



oncern.s between consistency and performance, and question of trading  choice to make highly depends on the circumstancesand may choose use explicit timeSynchronous vs. asynchronous, glitches, 
 
This structure can lead to specialized solutions (for example, the database community has known for decades how to solve the view-update problem for relational data). helps to implement efficient algorithms  to the view-update problem. Recursion is common, where views become subjects of other views: for example, signals in FRP can be defined in terms of other signals. Semantics can differ in how they treat 


use relations, may be monolithic, or graphs of objects. For example, database Facebook react has 
relationship, and how changes are propagated by the runtime. Prior approaches to reactive programming fall into two camps: explicit (the programmer expresses dependencies) and implicit (program expresses the viewdependencies are detected automatically, either at compile-time or at runtime).
Examples of explicit: makefiles, Rx, FRP. Examples of implicit: self-adjusting computation, Facebook React.


  is has focused on specifying dataflow explicity. However, there are some 


